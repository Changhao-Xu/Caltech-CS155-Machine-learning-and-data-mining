\newif\ifshowsolutions
\showsolutionstrue
\documentclass{article}
\usepackage{listings}
\usepackage{amsmath}
\usepackage{subfig}
\usepackage{amsthm}
\usepackage{amsmath}
\usepackage{amssymb}
\usepackage{graphicx}
\usepackage{mdwlist}
\usepackage{geometry}
\usepackage{titlesec}
\usepackage{palatino}
\usepackage{mathrsfs}
\usepackage{fancyhdr}
\usepackage{paralist}
\usepackage{todonotes}
\usepackage{tikz}
\usepackage{float} % Place figures where you ACTUALLY want it
\usepackage{comment} % A hack to toggle sections
\usepackage{ifthen}
\usepackage{mdframed}
\usepackage{verbatim}
\usepackage{listings}
\usepackage{bbm}
\usepackage{upquote} % Prevents backticks replacing single-quotes in verbatim
\usepackage[strings]{underscore}
\usepackage[colorlinks=true]{hyperref}
\usetikzlibrary{positioning,shapes,backgrounds}

\geometry{margin=1in}
\geometry{headheight=2in}
\geometry{top=2in}

\setlength{\marginparwidth}{2.15cm}
\setlength{\parindent}{0em}
\setlength{\parskip}{0.6\baselineskip}

\rhead{}
\lhead{}

% Spacing settings.
\titlespacing\section{0pt}{12pt plus 2pt minus 2pt}{0pt plus 2pt minus 2pt}
\titlespacing\subsection{0pt}{12pt plus 4pt minus 2pt}{0pt plus 2pt minus 2pt}
\titlespacing\subsubsection{0pt}{12pt plus 4pt minus 2pt}{0pt plus 2pt minus 2pt}
\renewcommand{\baselinestretch}{1.15}

% Shortcuts for commonly used operators.
\newcommand{\E}{\mathbb{E}}
\newcommand{\Var}{\operatorname{Var}}
\newcommand{\Cov}{\operatorname{Cov}}
\newcommand{\Bias}{\operatorname{Bias}}
\DeclareMathOperator{\argmin}{arg\,min}
\DeclareMathOperator{\argmax}{arg\,max}

% Do not number subsections and below.
\setcounter{secnumdepth}{1}

% Custom format subsection.
\titleformat*{\subsection}{\large\bfseries}

% Set up the problem environment.
\newcounter{problem}[section]
\newenvironment{problem}[1][]
  {\begingroup
    \setlength{\parskip}{0em}
    \refstepcounter{problem}\par\addvspace{1em}\textbf{Problem~\Alph{problem}\!
    \ifthenelse{\equal{#1}{}}{}{ [#1 points]}:}
  \endgroup}

% Set up the subproblem environment.
\newcounter{subproblem}[problem]
\newenvironment{subproblem}[1][]
  {\begingroup
    \setlength{\parskip}{0em}
    \refstepcounter{subproblem}\par\medskip\textbf{\roman{subproblem}.\!
    \ifthenelse{\equal{#1}{}}{}{ [#1 points]:}}
  \endgroup}

% Set up the teachers and materials commands.
\newcommand\teachers[1]
  {\begingroup
    \setlength{\parskip}{0em}
    \vspace{0.3em} \textit{\hspace*{2em} TAs responsible: #1} \par
  \endgroup}
\newcommand\materials[1]
  {\begingroup
    \setlength{\parskip}{0em}
    \textit{\hspace*{2em} Relevant materials: #1} \par \vspace{1em}
  \endgroup}

% Set up the hint environment.
\newenvironment{hint}[1][]
  {\begin{em}\textbf{Hint: }}
  {\end{em}}


% Set up the solution environment.
\ifshowsolutions
  \newenvironment{solution}[1][]
    {\par\medskip \begin{mdframed}\textbf{Solution~\Alph{problem}#1:} \begin{em}}
    {\end{em}\medskip\end{mdframed}\medskip}
  \newenvironment{subsolution}[1][]
    {\par\medskip \begin{mdframed}\textbf{Solution~\Alph{problem}#1.\roman{subproblem}:} \begin{em}}
    {\end{em}\medskip\end{mdframed}\medskip}
\else
  \excludecomment{solution}
  \excludecomment{subsolution}
\fi




%%%%%%%%%%%%%%%%%%%%%%%%%%%%%%
% HEADER
%%%%%%%%%%%%%%%%%%%%%%%%%%%%%%

\chead{
  {\vbox{
      \vspace{2mm}
      \large
      Machine Learning \& Data Mining \hfill
      Caltech CS/CNS/EE 155 \hfill \\[1pt]
      Set 4\hfill
      January 2020. \\
    }
  }
}

\begin{document}
\pagestyle{fancy}



%%%%%%%%%%%%%%%%%%%%%%%%%%%%%%
% PROBLEM 1
%%%%%%%%%%%%%%%%%%%%%%%%%%%%%%

\newpage
\section{Deep Learning Principles [35 Points]}
\materials{lectures on deep learning}

\begin{problem}[5]
  Backpropagation and Weight Initialization Part 1
\end{problem}

\begin{subsolution}

\end{subsolution}

\newpage

\begin{problem}[5]
  Backpropagation and Weight Initialization Part 2
\end{problem}

\begin{subsolution}

\end{subsolution}

\newpage


\problem \textbf{[10 Points]}


\begin{solution}

\end{solution}

\newpage




\problem Approximating Functions Part 1 \textbf{[7 Points]}

\begin{subsolution}

\end{subsolution}

\newpage


\problem Approximating Functions Part 2 \textbf{[8 Points]}

\begin{subsolution}

\end{subsolution}


% problem 2
\newpage
\section{Depth vs Width on the MNIST Dataset  [25 Points]}

\problem \textbf{Installation} \textbf{[2 Points]}

\begin{solution}

torch:

torchvision:

\end{solution}

\newpage



\problem \textbf{The Data} \textbf{[3 Points]}

\begin{subsolution}

\end{subsolution}

\newpage



 \problem \textbf{Modeling Part 1} \textbf{[8 Points]}

\begin{solution}

\end{solution}

\newpage


 \problem \textbf{Modeling Part 2} \textbf{[6 Points]}

 \begin{solution}

\end{solution}

\newpage


  \problem \textbf{Modeling Part 3} \textbf{[6 Points]}

  \begin{solution}

\end{solution}

 \newpage
 % problem 3
 \section{Convolutional Neural Networks  [40 Points]}
 \problem Zero Padding \textbf{[5 Points]}

\begin{solution}

\end{solution}

\newpage


\subsection{5 x 5 Convolutions}


\problem[2]

\begin{subsolution}

\end{subsolution}

\newpage


\problem[3]

\begin{subsolution}

\end{subsolution}

\newpage


 \subsection{Max/Average Pooling}

\problem[3]

\begin{subsolution}

\end{subsolution}

\newpage


\problem[3]

\begin{subsolution}

\end{subsolution}

\newpage


\problem[4]

\begin{subsolution}

\end{subsolution}

\newpage


\subsection{PyTorch implementation}
\problem[20]


\begin{subsolution}

\end{subsolution}

\end{document}
